%%%%%%%%%%%%%%%%%%%%%%%%%%%%%%%%%%%%%%%
% Deedy - One Page Two Column Resume
% LaTeX Template
% Version 1.2 (16/9/2014)
%
% Original author:
% Debarghya Das (http://debarghyadas.com)
%
% Original repository:
% https://github.com/deedydas/Deedy-Resume
%
% IMPORTANT: THIS TEMPLATE NEEDS TO BE COMPILED WITH XeLaTeX
%
% This template uses several fonts not included with Windows/Linux by
% default. If you get compilation errors saying a font is missing, find the line
% on which the font is used and either change it to a font included with your
% operating system or comment the line out to use the default font.
% 
%%%%%%%%%%%%%%%%%%%%%%%%%%%%%%%%%%%%%%
% 
% TODO:
% 1. Integrate biber/bibtex for article citation under publications.
% 2. Figure out a smoother way for the document to flow onto the next page.
% 3. Add styling information for a "Projects/Hacks" section.
% 4. Add location/address information
% 5. Merge OpenFont and MacFonts as a single sty with options.
% 
%%%%%%%%%%%%%%%%%%%%%%%%%%%%%%%%%%%%%%
%
% CHANGELOG:
% v1.1:
% 1. Fixed several compilation bugs with \renewcommand
% 2. Got Open-source fonts (Windows/Linux support)
% 3. Added Last Updated
% 4. Move Title styling into .sty
% 5. Commented .sty file.
%
%%%%%%%%%%%%%%%%%%%%%%%%%%%%%%%%%%%%%%%
%
% Known Issues:
% 1. Overflows onto second page if any column's contents are more than the
% vertical limit
% 2. Hacky space on the first bullet point on the second column.
%
%%%%%%%%%%%%%%%%%%%%%%%%%%%%%%%%%%%%%%


\documentclass[]{resume}
\usepackage{fancyhdr}

\pagestyle{fancy}
\fancyhf{}
 
\begin{document}

%%%%%%%%%%%%%%%%%%%%%%%%%%%%%%%%%%%%%%
%
%     LAST UPDATED DATE
%
%%%%%%%%%%%%%%%%%%%%%%%%%%%%%%%%%%%%%%
\lastupdated

%%%%%%%%%%%%%%%%%%%%%%%%%%%%%%%%%%%%%%
%
%     TITLE NAME
%
%%%%%%%%%%%%%%%%%%%%%%%%%%%%%%%%%%%%%%
\namesection{Rahul}{Ghosh}{ %\urlstyle{same}\href{http://debarghyadas.com}{debarghyadas.com}| \href{http://fb.co/dd}{fb.co/dd}\\
\href{mailto:rahulghosh2021@gmail.com}{rahulghosh2021@gmail.com} | 8420784442 | \href{mailto:rahul.ghosh@samsung.com}{rahul.ghosh@samsung.com}
}

%%%%%%%%%%%%%%%%%%%%%%%%%%%%%%%%%%%%%%
%
%     COLUMN ONE
%
%%%%%%%%%%%%%%%%%%%%%%%%%%%%%%%%%%%%%%

\begin{minipage}[t]{0.33\textwidth} 

%%%%%%%%%%%%%%%%%%%%%%%%%%%%%%%%%%%%%%
%     EDUCATION
%%%%%%%%%%%%%%%%%%%%%%%%%%%%%%%%%%%%%%

\section{Education} 

\subsection{Indian Institute of \newline Technology, Guwahati}
\descript{B.Tech in Electronics \& Electrical Engineering}
\location{CGPA: 7.8/10 \\ May 2016 | Guwahati, India}
\sectionsep

%\subsection{Cornell University}
%\descript{BS in Computer Science}
%\location{May 2014 | Ithaca, NY}
%College of Engineering \\
%Magna Cum Laude\\
%\location{ Cum. GPA: 3.83 / 4.0 \\
%Major GPA: 3.9 / 4.0}
%\sectionsep

\subsection{The Army Public School}
\descript{Higher Secondary Examination}
\location{Percentage: 92.6\% \\ Grad. May 2011 | Pune, India}
\descript{Secondary School Certificate}
\location{Percentage: 93.8\% \\ Grad. May 2009 | Pune, India}
\sectionsep

%%%%%%%%%%%%%%%%%%%%%%%%%%%%%%%%%%%%%%
%     LINKS
%%%%%%%%%%%%%%%%%%%%%%%%%%%%%%%%%%%%%%

\section{Links} 
Github:// \href{https://github.com/2021rahul}{\bf 2021rahul} \\
LinkedIn://  \href{https://www.linkedin.com/in/2021rahul}{\bf 2021rahul} \\
Facebook:// \href{https://www.facebook.com/2021rahul}{\bf 2021rahul} \\
Twitter://  \href{https://twitter.com/2021rahul}{\bf @2021rahul} \\
%YouTube://  \href{https://www.youtube.com/user/DeedyDash007}{\bf DeedyDash007} \\
%Quora://  \href{https://www.quora.com/Debarghya-Das}{\bf Debarghya-Das}

%%%%%%%%%%%%%%%%%%%%%%%%%%%%%%%%%%%%%%
%     COURSEWORK
%%%%%%%%%%%%%%%%%%%%%%%%%%%%%%%%%%%%%%

\section{Coursework}
% \subsection{Graduate}
% Advanced Machine Learning \\
% Open Source Software Engineering \\
% Advanced Interactive Graphics \\
% Compilers + Practicum \\
% Cloud Computing \\
% Evolutionary Computation \\
% Defending Computer Networks \\
% Machine Learning \\
% \sectionsep

\subsection{Undergraduate}
Embedded Systems \& Computer Architecture\\
Pattern Recognition \& Machine Learning \\
Digital Image Processing \\
Computer Vision \\
Probability and Random Processes \\
Parallel Computing\\
Introduction to Computing {\footnotesize \textit{\textbf{(theory \& laboratory course) }}} \\
Digital Circuits and Microprocessors {\footnotesize \textit{\textbf{(theory \& laboratory course) }}} \\
Algorithms and Data Structure \\
Mathematics {\footnotesize \textit{\textbf{(Linear Algebra, Real Analysis, Multivariable Calculus, Complex Analysis) }}} \\
Data Scientist Toolbox and R Programming {\footnotesize \textit{\textbf{(Coursera) }}} \\



% Information Retrieval \\
% Operating Systems \\
% Artificial Intelligence + Practicum \\
% Functional Programming \\
% Computer Graphics + Practicum \\
% {\footnotesize \textit{\textbf{(Research Asst. \& Teaching Asst 2x) }}} \\
% Unix Tools and Scripting \\

%%%%%%%%%%%%%%%%%%%%%%%%%%%%%%%%%%%%%%
%     SKILLS
%%%%%%%%%%%%%%%%%%%%%%%%%%%%%%%%%%%%%%

\section{Skills}
\subsection{Programming}
C \textbullet{} C++ \textbullet{} R \textbullet{} Python \textbullet{} Ruby on Rails \textbullet{} Shell \textbullet{} 8051 \& 8085 Assembly
\subsection{Tools/Frameworks}
Theano \textbullet{} TensorFlow \textbullet{} OpenCV \textbullet{} MATLAB \textbullet{} \LaTeX\ \textbullet{} Multisim \textbullet{} Arduino Uno Programming
\subsection{Operating Systems}
Linux (Ubuntu) \textbullet{} Mac OS \textbullet{} Windows
% \location{Over 5000 lines:}
% Java \textbullet{}   Shell \textbullet{} Python \textbullet{} Javascript \\
% OCaml \textbullet{} Matlab \textbullet{} Rails \textbullet{} \LaTeX\ \\ 
% \location{Over 1000 lines:}
% C \textbullet{} C++ \textbullet{} CSS \textbullet{} PHP \textbullet{} Assembly \\
% \location{Familiar:}
% AS3 \textbullet{} iOS \textbullet{} Android \textbullet{} MySQL
\sectionsep

%%%%%%%%%%%%%%%%%%%%%%%%%%%%%%%%%%%%%%
%
%     COLUMN TWO
%
%%%%%%%%%%%%%%%%%%%%%%%%%%%%%%%%%%%%%%

\end{minipage} 
\hfill
\begin{minipage}[t]{0.66\textwidth} 

%%%%%%%%%%%%%%%%%%%%%%%%%%%%%%%%%%%%%%
%     EXPERIENCE
%%%%%%%%%%%%%%%%%%%%%%%%%%%%%%%%%%%%%%

\section{Experience}
\runsubsection{Samsung R\&D}
\descript{| Software Engineer }
\location{Samsung Research Institute - Delhi}
\location{July 2016 - Present | Noida, India}
\sectionsep

\runsubsection{IDRBT {\footnotesize ESTD. BY Reserve Bank of India}}
\descript{| Project Trainee}
\location{Institute of Development \& Research in Banking Technology}
\location{May 2015 – July 2015 | Hyderabad, India}
\vspace{\topsep} % Hacky fix for awkward extra vertical space
\begin{tightemize}
\item Worked with \textbf{\href{http://www.idrbt.ac.in/vravi.html}{Dr. Vadlamani Ravi}} and completed 2 projects.
\item Co-authored 2 research papers based on the projects carried out during the internship period.
\item Completed MATLAB software certification from the institute.
% \item 52 out of 2500 applicants chosen to be a KPCB Fellow 2014.
% \item Led and shipped Yoda - the admin interface for the new Phoenix platform. 
% \item Full-stack developer - Wrote and reviewed code for JS using Backbone, Jade, Stylus and Require and Scala using Play
\end{tightemize}
\sectionsep

% \runsubsection{Google}
% \descript{| Software Engineering Intern }
% \location{May 2013 – Aug 2013 | Mountain View, CA}
% \begin{tightemize}
% \item Worked on the YouTube Captions team, in Javascript and Python to plan, to design and develop the full stack to add and edit Automatic Speech Recognition captions. In production.
% \item Created a backbone.js-like framework for the Captions editor.
% \end{tightemize}
% \sectionsep

% \runsubsection{Phabricator}
% \descript{| Open Source Contributor \& Team Leader}
% \location{Jan 2013 – May 2013 | Palo Alto, CA \& Ithaca, NY}
% \begin{tightemize}
% \item Phabricator is used daily by Facebook, Dropbox, Quora, Asana and more.
% \item I created the Meme generator and more in PHP and Shell.
% \item Led a team from MIT, Cornell, IC London and UHelsinki for the project.
% \end{tightemize}
% \sectionsep

%%%%%%%%%%%%%%%%%%%%%%%%%%%%%%%%%%%%%%
%     RESEARCH
%%%%%%%%%%%%%%%%%%%%%%%%%%%%%%%%%%%%%%

\section{Projects}
\runsubsection{\href{https://github.com/2021rahul/Bachelor_Thesis_Project}{GLAND SEGMENTATION \& CLASSIFICATION USING DEEP LEARNING}}
\descript{| BACHELOR THESIS PROJECT}
\location{AUG 2015 – APR 2016 | IIT Guwahati, India}
Worked with \textbf{\href{http://www.iitg.ernet.in/eee/amitsethi.html}{Prof Amit Sethi}} to create a novel approach to segment H\&E stained colon tissue images of the Warwick-QU dataset from the gland segmentation contest at MICCAI'15 by using Convolutional Neural Networks.
\sectionsep

\runsubsection{Pattern Recognition \& Machine Learning Course Project}\\
\location{AUG 2014 – APR 2015 | IIT Guwahati, India}
Designed and Implemented projects such as Handwritten Digit Recognition, E-mail Classifier and Face Recognition etc under the guidance of \textbf{\href{http://www.iitg.ernet.in/eee/suresh.html}{Prof Suresh Sundaram}}
\sectionsep

\runsubsection{\href{http://github.com/ashimajain2595/ComputerVision_Project}{PCB Fault Detection \& Classification}}\\
\location{MAR 2016 – APR 2016 | IIT Guwahati, India}
Developed an automated system for PCB fault detection and classification under the guidance of \textbf{\href{http://www.iitg.ernet.in/eee/mkb.html}{Prof M.K. Bhuyan}} 
\sectionsep

\runsubsection{WALL CLIMBING ROBOT}
\descript{| DESIGN COURSE PROJECT}
\location{JAN 2015 – APR 2015 | IIT Guwahati, India}
Worked with \textbf{\href{http://www.iitg.ernet.in/eee/pguha.html}{Prof Prithwijit Guha}} to develop a wall climbing robot with payload capable of climbing any rugged terrain.
\sectionsep

%%%%%%%%%%%%%%%%%%%%%%%%%%%%%%%%%%%%%%
%     ACHIEVEMENTS
%%%%%%%%%%%%%%%%%%%%%%%%%%%%%%%%%%%%%%

\section{Achievements \& Test Scores} 
\begin{tabular}{rll}
2016         & 327/340 & Graduate Record Examinations(GRE)\\
2012	     & 99.6 percentile  & IIT-Joint Entrance Exam(IIT-JEE)\\
2012	     & 99.87 percentile  & All India Engineering Entrance Exam(AIEEE)\\
2012	     & All India Rank of 534  & ISAT-2012\\
2011     & All India Rank of 6 & International Master Mathematics Olympiad\\
% 2011     & Special Prize & Highest marks in Maths and above 90\% marks\\ 
%          & &in senior secondary examination\\
% 2011     & National & Indian National Mathematics Olympiad (INMO) Finalist \\
\end{tabular}
\sectionsep

%%%%%%%%%%%%%%%%%%%%%%%%%%%%%%%%%%%%%%
%     PUBLICATIONS
%%%%%%%%%%%%%%%%%%%%%%%%%%%%%%%%%%%%%%

\section{Publications}
\vspace{\topsep}
\vspace{\topsep}
% \renewcommand\refname{\vskip 3cm} % Couldn't get this working from the .cls file
\bibliographystyle{abbrv}
\bibliography{publications}
\nocite{*}
\sectionsep

\end{minipage} 
\end{document}  \documentclass[]{article}
